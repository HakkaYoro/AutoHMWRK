
\documentclass[12pt, spanish]{article}
\usepackage[utf8]{inputenc}
\usepackage[spanish]{babel}
\usepackage{graphicx}
\usepackage{geometry}
\geometry{a4paper, margin=2.5cm}

\begin{document}

\begin{tabular}{ll}
Nombre: & Gabriel Méndez \\
C.I.: & V-30.464.232 \\
Materia: & Análisis de Sistemas \\
Sección: & DCM0402 \\
Universidad: & UniversidadAlejandroDeHumboldt \\
Carrera: & IngenieriaInformatica \\
Evaluación: & Evaluación N\textdegree1 – Corte 1 \\
Fecha: & 11/02/2025
\end{tabular}

\vspace{2cm}
\begin{center}
\Large \textbf{{Materia: Análisis de Sistemas}}
\end{center}

\section{Introducción}

El desarrollo de software ha evolucionado significativamente en las últimas décadas, dando lugar a una variedad de metodologías, cada una con sus propias fortalezas y debilidades. Las metodologías tradicionales, como el modelo en cascada y el modelo en V, ofrecen una estructura bien definida y un control riguroso, pero a menudo carecen de la flexibilidad necesaria para adaptarse a los cambios rápidos y a los requisitos volátiles. Por otro lado, las metodologías ágiles, como Scrum y Extreme Programming (XP), priorizan la adaptabilidad, la colaboración y la entrega iterativa, pero pueden carecer de la documentación exhaustiva y la planificación a largo plazo que requieren algunos proyectos.  Este documento propone un modelo híbrido de desarrollo de software que combina elementos de metodologías ágiles y tradicionales, buscando maximizar sus respectivas ventajas y mitigar sus desventajas.

\section{Modelo Híbrido de Desarrollo de Software}

El modelo propuesto se basa en la idea de dividir el ciclo de vida del desarrollo de software en fases que se benefician de diferentes enfoques metodológicos.  El modelo se puede resumir en las siguientes fases:

\begin{enumerate}
  \item \textbf{Fase de Planificación Inicial (Tradicional):} Esta fase, inspirada en el modelo en cascada, se centra en definir los requisitos iniciales del proyecto, establecer un alcance claro y crear una arquitectura general del sistema. Se realiza un análisis detallado de los riesgos y se crea un plan de proyecto inicial. El resultado de esta fase es un documento de requisitos de alto nivel y un plan de proyecto preliminar.
  \item \textbf{Fase de Desarrollo Iterativo (Ágil):}  Esta fase utiliza Scrum o XP para desarrollar el software de forma iterativa e incremental.  El equipo trabaja en sprints cortos (por ejemplo, de dos semanas) para entregar funcionalidades concretas. Se fomenta la colaboración continua entre los miembros del equipo y con el cliente (o representante del cliente). Se realizan revisiones al final de cada sprint para obtener retroalimentación y adaptar el plan según sea necesario. Se utiliza la integración continua y las pruebas automatizadas para garantizar la calidad del software.
  \item \textbf{Fase de Integración y Pruebas (Tradicional con elementos Ágiles):} Una vez que se han desarrollado suficientes iteraciones, se realiza una fase de integración y pruebas más exhaustiva.  Esta fase sigue los principios del modelo en V, con pruebas unitarias, de integración, de sistema y de aceptación. Sin embargo, a diferencia del modelo en V tradicional, las pruebas se realizan de forma continua durante el desarrollo iterativo, lo que permite identificar y corregir los defectos de forma temprana.  Además, la retroalimentación obtenida durante las pruebas puede llevar a nuevas iteraciones de desarrollo.
  \item \textbf{Fase de Despliegue y Mantenimiento (Tradicional con elementos Ágiles):}  Una vez que el software ha pasado las pruebas de aceptación, se despliega en el entorno de producción. El mantenimiento se realiza de forma continua, utilizando los principios de DevOps para automatizar los procesos de despliegue y monitorización.  La retroalimentación de los usuarios se utiliza para planificar futuras mejoras y actualizaciones del software.  Pequeños cambios y correcciones se realizan siguiendo metodologías ágiles, mientras que grandes actualizaciones pueden requerir un nuevo ciclo de desarrollo.
\end{enumerate}

\subsection{Ejemplo Práctico}

Consideremos el desarrollo de una aplicación web para una tienda online.

\begin{itemize}
    \item \textbf{Planificación Inicial:} Se definen los requisitos principales: catálogo de productos, carrito de compras, proceso de pago, gestión de usuarios. Se establece la arquitectura general: front-end (React), back-end (Node.js), base de datos (MongoDB).
    \item \textbf{Desarrollo Iterativo (Scrum):} El equipo trabaja en sprints de dos semanas.  Sprint 1:  desarrollo del catálogo de productos.  Sprint 2: desarrollo del carrito de compras.  Sprint 3: desarrollo del proceso de pago (integración con una pasarela de pago).
    \item \textbf{Integración y Pruebas:} Se realizan pruebas de integración entre el front-end y el back-end, y pruebas de sistema para verificar el correcto funcionamiento de la aplicación.
    \item \textbf{Despliegue y Mantenimiento:}  La aplicación se despliega en un servidor de producción.  Se monitoriza el rendimiento de la aplicación y se realizan correcciones de errores según sea necesario.  Posteriormente, se añade la funcionalidad de gestión de descuentos mediante un nuevo sprint.
\end{itemize}

\section{Ventajas del Modelo Híbrido}

*   \textbf{Adaptabilidad:} El modelo permite adaptarse a los cambios en los requisitos y las prioridades del cliente gracias a la flexibilidad de las metodologías ágiles.
*   \textbf{Control:} La fase de planificación inicial proporciona un control riguroso sobre el alcance y la arquitectura del proyecto.
*   \textbf{Calidad:} Las pruebas continuas y la integración frecuente garantizan la calidad del software.
*   \textbf{Entrega temprana de valor:}  La entrega iterativa permite entregar funcionalidades concretas de forma temprana, lo que permite al cliente obtener valor desde el principio del proyecto.
*   \textbf{Mitigación de riesgos:} El análisis de riesgos inicial y la gestión iterativa de riesgos permiten mitigar los riesgos de forma proactiva.

\section{Desafíos del Modelo Híbrido}

*   \textbf{Complejidad:} La combinación de metodologías ágiles y tradicionales puede aumentar la complejidad del proceso de desarrollo.
*   \textbf{Comunicación:} La comunicación entre los miembros del equipo y con el cliente es fundamental para el éxito del modelo.
*   \textbf{Cultura:** Se requiere una cultura de colaboración y aprendizaje continuo para que el modelo funcione eficazmente.
*   \textbf{Selección de la metodología adecuada:** Elegir la metodología ágil (Scrum vs XP) para cada iteración puede ser un desafío.
*   \textbf{Resistencia al cambio:**  La transición de metodologías tradicionales a híbridas puede encontrar resistencia por parte de los miembros del equipo acostumbrados a un enfoque más estructurado.

\section{Conclusión}

El modelo híbrido de desarrollo de software propuesto ofrece un enfoque equilibrado que combina las fortalezas de las metodologías ágiles y tradicionales.  Si bien presenta algunos desafíos, las ventajas en términos de adaptabilidad, control, calidad y entrega temprana de valor lo convierten en una opción atractiva para muchos proyectos de software.  La clave para el éxito del modelo radica en una planificación cuidadosa, una comunicación efectiva y una cultura de colaboración y aprendizaje continuo. La selección de la metodología ágil específica (Scrum, XP, Kanban) dentro de la fase iterativa debe estar basada en las necesidades y características del proyecto.

\begin{thebibliography}{9}
\bibitem{sommerville} Sommerville, I. (2016). \textit{Software engineering} (10th ed.). Pearson Education.
\bibitem{pressman} Pressman, R. S., & Maxim, B. R. (2015). \textit{Software engineering: A practitioner's approach} (8th ed.). McGraw-Hill Education.
\bibitem{beck} Beck, K. (2000). \textit{Extreme programming explained: Embrace change}. Addison-Wesley Professional.
\bibitem{schwaber} Schwaber, K., & Sutherland, J. (2020). \textit{The Scrum guide}. Scrum.org. Recuperado de https://www.scrum.org/
\end{thebibliography}

\end{document}
