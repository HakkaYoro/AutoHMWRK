\documentclass[12pt, spanish]{article}
\usepackage[utf8]{inputenc}
\usepackage[spanish]{babel}
\usepackage{graphicx}
\usepackage{geometry}
\geometry{a4paper, margin=2.5cm}

\begin{document}

\begin{tabular}{ll}
Nombre: & Gabriel Méndez \\
C.I.: & V-30.464.232 \\
Materia: & Análisis de Sistemas \\
Sección: & DCM0402 \\
Universidad: & UniversidadAlejandroDeHumboldt \\
Carrera: & IngenieriaInformatica \\
Evaluación: & Evaluación N\textdegree1 – Corte 1 \\
Fecha: & 11/02/2025
\end{tabular}

\vspace{2cm}
\begin{center}
\Large \textbf{{Materia: Análisis de Sistemas}}
\end{center}

\section*{Introducción}

El desarrollo de software moderno se enfrenta a desafíos constantes debido a la rápida evolución de las tecnologías y las necesidades cambiantes de los usuarios. Si bien las metodologías ágiles han ganado popularidad por su flexibilidad y capacidad de adaptación, las metodologías tradicionales aún conservan su valor en ciertos contextos. Este documento propone un modelo híbrido que integra los beneficios de las metodologías ágiles y tradicionales, buscando optimizar el proceso de desarrollo de software en función de las características específicas del proyecto. El objetivo es crear un enfoque más robusto y adaptable que pueda abordar una variedad de escenarios y maximizar las posibilidades de éxito del proyecto.

\section*{Desarrollo del Modelo Híbrido}

Nuestro modelo híbrido, al que llamaremos "Agile-V", se basa en la combinación de Scrum (una metodología ágil) con el Modelo V (una metodología tradicional). La idea principal es utilizar la estructura del Modelo V para definir las fases generales del proyecto y los entregables principales, mientras que dentro de cada fase se implementa Scrum para la gestión del desarrollo iterativo e incremental.

**Fases del Modelo Agile-V:**

1.  **Análisis de Requisitos (Modelo V):** Esta fase se alinea con la primera fase del Modelo V. Se recopilan y documentan los requisitos del cliente de manera exhaustiva. Se crea un documento de requisitos detallado que servirá como base para las siguientes fases. En esta fase, se puede usar técnicas tradicionales como entrevistas y talleres estructurados.

2.  **Diseño del Sistema (Modelo V):**  Se desarrolla la arquitectura general del sistema y se definen los componentes principales. Se elaboran diagramas de arquitectura y se especifican las interfaces entre los componentes. Al igual que la fase anterior, se apega a la estructura tradicional del Modelo V.

3.  **Sprint 0 - Planificación (Scrum):**  Antes de iniciar el desarrollo iterativo, se realiza un "Sprint 0" para planificar los Sprints subsiguientes.  Se define el Product Backlog inicial, se priorizan las funcionalidades y se estima el esfuerzo necesario. Se identifican los riesgos y se definen las estrategias de mitigación.

4.  **Desarrollo Iterativo (Scrum integrado al Modelo V):** Esta fase comprende una serie de Sprints. Cada Sprint tiene una duración fija (por ejemplo, 2-4 semanas). En cada Sprint, el equipo selecciona funcionalidades del Product Backlog y las desarrolla, prueba e integra. Al final de cada Sprint, se presenta una versión funcional del software al Product Owner para su revisión y feedback.  Esta fase se repite varias veces, cubriendo los niveles de diseño detallado, implementación y pruebas unitarias del Modelo V, pero con el dinamismo de Scrum. Las pruebas de integración se realizan al final de cada sprint sobre las funcionalidades implementadas.

5.  **Pruebas del Sistema (Modelo V):** Una vez finalizados los Sprints, se realizan pruebas de sistema exhaustivas para verificar que el software cumple con todos los requisitos especificados. Se documentan los resultados de las pruebas y se corrigen los defectos encontrados.

6.  **Pruebas de Aceptación (Modelo V):**  El cliente realiza pruebas de aceptación para verificar que el software cumple con sus expectativas y necesidades. Se documentan los resultados de las pruebas y se corrigen los defectos encontrados.

7.  **Implementación y Mantenimiento (Modelo V):** El software se implementa en el entorno de producción y se brinda soporte y mantenimiento continuo. Se realizan actualizaciones y correcciones de errores según sea necesario.

**Roles:**

*   **Product Owner:** Responsable de definir y priorizar el Product Backlog. Interactúa con los stakeholders para entender sus necesidades y asegurarse de que el software cumple con sus expectativas.
*   **Scrum Master:** Facilita el proceso Scrum, eliminando impedimentos y asegurándose de que el equipo sigue las prácticas ágiles.
*   **Equipo de Desarrollo:** Responsable de desarrollar, probar e integrar el software en cada Sprint.
*   **Analista de Negocio:**  Responsable de la recopilación y documentación de los requisitos en las fases iniciales. Su rol se atenúa durante la fase iterativa, pero sigue presente para aclarar dudas y ajustar requisitos.
*   **Arquitecto de Software:** Responsable de definir la arquitectura del sistema en la fase de diseño.  Su participación se centra en las fases iniciales, pero se le consulta durante el desarrollo iterativo para garantizar la coherencia de la arquitectura.

**Ejemplo Práctico:**

Consideremos el desarrollo de una aplicación web para una tienda en línea. La fase de Análisis de Requisitos definiría las funcionalidades principales (catálogo de productos, carrito de compras, proceso de pago, gestión de usuarios, etc.). La fase de Diseño del Sistema establecería la arquitectura de la aplicación (capas, bases de datos, etc.). Luego, los Sprints se enfocarían en implementar incrementalmente cada funcionalidad, permitiendo al cliente revisar y probar cada incremento a medida que se desarrolla.  Por ejemplo, el Sprint 1 podría implementar la visualización del catálogo de productos, el Sprint 2 el carrito de compras, y así sucesivamente.  Las pruebas del sistema validarían que todas las funcionalidades trabajen juntas correctamente, y las pruebas de aceptación asegurarían que la aplicación cumpla con las expectativas del cliente.

\section*{Ventajas del Modelo Agile-V}

*   **Mejor Gestión de Riesgos:** La estructura del Modelo V permite identificar y mitigar los riesgos en las fases iniciales del proyecto.
*   **Mayor Flexibilidad:** El uso de Scrum dentro de cada fase permite adaptarse a los cambios en los requisitos y prioridades.
*   **Entrega Continua de Valor:** Los Sprints permiten entregar versiones funcionales del software al cliente en intervalos regulares.
*   **Mayor Transparencia:** El proceso Scrum promueve la transparencia y la comunicación entre el equipo y el cliente.
*   **Adecuado para Proyectos Complejos:** La combinación de la estructura del Modelo V con la flexibilidad de Scrum lo hace adecuado para proyectos complejos con requisitos bien definidos pero que también requieren adaptabilidad.
*   **Reducción de retrabajos:** La validación continua en cada sprint reduce la probabilidad de errores costosos en las etapas finales.

\section*{Desafíos del Modelo Agile-V}

*   **Complejidad de la Integración:** La integración de Scrum con el Modelo V puede ser compleja y requiere una planificación cuidadosa.
*   **Curva de Aprendizaje:** El equipo debe estar capacitado en metodologías ágiles y tradicionales.
*   **Mayor Comunicación:** Se requiere una comunicación constante y efectiva entre el equipo, el Product Owner y el cliente.
*   **Dificultad en la Estimación:** Estimar el esfuerzo necesario para cada Sprint puede ser difícil, especialmente en las fases iniciales del proyecto.
*   **Potencial de conflictos:** La combinación de filosofías diferentes puede generar conflictos entre los miembros del equipo si no se gestiona adecuadamente.
*   **Sobrecarga de documentación:** La combinación del rigor del Modelo V con la agilidad de Scrum puede resultar en una sobrecarga de documentación si no se equilibra correctamente.

\section*{Conclusión}

El modelo Agile-V ofrece un enfoque prometedor para el desarrollo de software, combinando los beneficios de las metodologías ágiles y tradicionales. Si bien presenta desafíos, su capacidad para gestionar riesgos, adaptarse a los cambios y entregar valor de forma continua lo convierte en una opción viable para una amplia gama de proyectos. La clave para el éxito reside en una planificación cuidadosa, una comunicación efectiva y un equipo capacitado. La adopción de este modelo híbrido, con la debida adaptación al contexto específico de cada proyecto, puede significativamente mejorar la calidad del software y la satisfacción del cliente.

\section*{Referencias}

*   Schwaber, K., & Sutherland, J. (2020). *The Scrum Guide*. Scrum.org.
*   Sommerville, I. (2016). *Software Engineering* (10th ed.). Pearson Education.
*   Pressman, R. S., & Maxim, B. R. (2015). *Software Engineering: A Practitioner's Approach* (8th ed.). McGraw-Hill Education.
*   Larman, C., & Basili, V. R. (2003). Iterative and Incremental Development: A Brief History. *IEEE Computer*, *36*(6), 47-56.
*   Royce, W. W. (1970). Managing the Development of Large Software Systems. *Proceedings of IEEE WESCON*, 1-9.
\end{document}
